%
% Copyright (C) 2004-2009 Jason Blevins <jrblevin@sdf.lonestar.org>
% http://jblevins.org/projects/cv-template/
%
% You may use use this document as a template to create your own CV
% and you may redistribute the source code freely. No attribution is
% required in any resulting documents. I do ask that you please leave
% this notice and the above URL in the source code if you choose to
% redistribute this file.

\documentclass[letterpaper, 12pt]{article}

\usepackage{hyperref}
\usepackage{geometry}

\usepackage[spanish]{babel}
  \selectlanguage{spanish}

% Comment the following lines to use the default Computer Modern font
% instead of the Palatino font provided by the mathpazo package.
% Remove the 'osf' bit if you don't like the old style figures.
%\usepackage[T1]{fontenc}
\usepackage[utf8x]{inputenc}
%\usepackage[sc,osf]{mathpazo}
\usepackage[scaled]{uarial}
\renewcommand*\familydefault{\sfdefault} %% Only if the base font of the document is to be sans serif
\usepackage[T1]{fontenc}

% Set your name here
\def\name{George Gerald Vega Yon}

% Replace this with a link to your CV if you like, or set it empty
% (as in \def\footerlink{}) to remove the link in the footer:
\def\footerlink{http://cl.linkedin.com/in/georgevegayon}

% The following metadata will show up in the PDF properties
\hypersetup{
  colorlinks = true,
  urlcolor = blue,
  pdfauthor = {\name},
  pdfkeywords = {economics, statistics, mathematics},
  pdftitle = {\name: Curriculum Vitae},
  pdfsubject = {Curriculum Vitae},
  pdfpagemode = UseNone
}

\geometry{
  body={6.5in, 9in},
  left=1.0in,
  top=1.25in
}

% Customize page headers
\pagestyle{myheadings}
\markright{\name}
\thispagestyle{empty}

% Custom section fonts
%\usepackage{sectsty}
%\sectionfont{\sffamily\mdseries\Large}
%\subsectionfont{\sffamily\mdseries\itshape\large}

% Other possible font commands include:
% rmfamily
% \ttfamily for teletype,
% \sffamily for sans serif,
% \bfseries for bold,
% \scshape for small caps,
% \normalsize, \large, \Large, \LARGE sizes.

% Don't indent paragraphs.
\setlength\parindent{0em}

% Make lists without bullets
\renewenvironment{itemize}{
  \begin{list}{}{
    \setlength{\leftmargin}{0.45cm}
  }
}{
  \end{list}
}

% Para poder poner comandos genericos en tablas (en el inicio del argumento)
\usepackage{array}

\usepackage{setspace}
  \onehalfspace

\begin{document}

% Place name at left
{\huge \name}

% Alternatively, print name centered and bold:
%\centerline{\huge \bf \name}

\vspace{0.25in}

\begin{minipage}{0.45\linewidth}
  \begin{tabular}{>{\bfseries}p{4cm}l}
    Teléfono Móvil & +56 (9) 7 647 2552 \\
    Correo electrónico & \href{mailto:g.vegayon@gmail.com}{\tt g.vegayon@gmail.com} \\
   Sitio Web & \href{http://cl.linkedin.com/in/georgevegayon}{\tt http://cl.linkedin.com/in/georgevegayon} \\
  \end{tabular}
\end{minipage}

\section*{Datos Personales}

\begin{minipage}{0.45\linewidth}
  \begin{tabular}{>{\bfseries}p{4cm}l}
  RUT & 16864739-5\\
  Fecha de Nacimiento & 2 de marzo de 1988.\\
  Nacionalidad & Chileno\\
  Estado civil & Casado\\
  Residencia & Santiago Centro, Región Metropolitana.
  \end{tabular}
\end{minipage}


\section*{Educación}

\begin{itemize}
\item Magíster en Economía y Políticas Públicas, Universidad Adolfo Ibáñez (2010)
\item Ingeniería Comercial, Universidad Adolfo Ibáñez, \emph{Beca de Excelencia Académica} (2010)
\item Minor en Cs. Políticas, Universidad Adolfo Ibáñez (2009)
\end{itemize}

\section*{Publicaciones}
\subsection*{Componentes de Software}
\begin{itemize}
\item ``rgexf: An R library to build gexf graph files'' en {\it Comprehensive R Archive Network}, 2012
\item ``googlePublicData: An R library to build Google's Public Data Explorer DSPL Metadata files'' en {\it Comprehensive R Archive Network}, 2012
\item ``CUENTACOT: Stata module for Contributions Counting'' en {\it Statistical Software Components, Boston College Department of Economics}, 2011
\item ``MOVAVG: MATA based Moving Average generator'' en {\it Statistical Software Components, Boston College Department of Economics}, 2012
\item ``DOPARSER: Look for dta-files used in do-files|ado-files (plain-text)'' en {\it Statistical Software Components, Boston College Department of Economics}, 2012
\end{itemize}

\subsection*{Trabajo en Progreso}
\begin{itemize}
\item ``PIB Académico: Dinámicas de la colaboración académica en Chile'' en {\it Universidad Adolfo Ibáñez}
\item ``Discarding variables: A practical application of Jolliffe's (1972) B2 algorithm for subset selection using Principal Components Analysis'' en {\it Superintendencia de Pensiones}.
\item ``Formación de la Opinión Pública en los Medios Sociales'' en {\it Universidad Adolfo Ibáñez}.
\end{itemize}

\subsection*{Prensa}

\section*{Experiencia Laboral}

\begin{itemize}
\item \textbf{Ministerio de Planificación, 2011.} \emph{Analista de la Unidad de Monitoreo de Programas Sociales} coordinando proyecto de desarrollo del \emph{Banco Integrado de Programas Sociales}--referenciado en Proyecto de Ley del \emph{Ministerio de Desarrollo Social}-- y encargado de estrategia de \emph{datos abiertos} en la división.
\item \textbf{Universidad Adolfo Ibáñez, 2011.} \emph{Profesor Auxiliar} en curso \emph{Introducción a la Economía} para alumnos de primer año de Ingeniería Comercial.
\item \textbf{Ministerio de Educación de Chile, 2009.} \emph{Asesor de Jefe de Servicio}, PhD. Rodrigo Fábrega Lacoa, en la \emph{Oficina de Relaciones Internacionales} y en el \emph{Programa Inglés Abre Puertas} en desarrollo de Sistemas de Gestión de Bases de Datos Relacionales y en Análisis de Datos.
\end{itemize}


\section*{Actividades Extracurriculares}

\begin{itemize}
\item \textbf{Proyecto UAIEduca , 2006--2007.} \emph{Voluntario} como tutor para alumnos de 4$\circ$ año medio de colegio municipal en Peñalolén, \emph{Centro Educacional Eduardo de la Barra}, en módulo de matemáticas. Año inmediato a mi incorporación asumí como coordinador del proyecto.
\item \textbf{Cursos Microeconomía II (UMAYOR) y Microeconomía (UAI), 2007--2008.} \emph{Ayudante} de ambos cursos semestrales dictados por el profesor M. Sc. Matías Petersen Cortés.
\item \textbf{Centro de Alumnos Escuela de Negocios Santiago, 2008.} \emph{Directivo y Coordinador General} del CAA a cargo de la planificación anual de actividades y coordinación de los distintos proyectos desarrollados durante el 2008.
\item \textbf{Proyecto Educación Joven, 2008--2009.} \emph{Director y co-fundador} de programa de educación social que, con un equipo de 20 voluntarios, benefició a 70 alumnos de distintos establecimientos educacionales de la Región Metropolitana. Fin del programa da inicio a programa social \href{http://www.preujoven.cl}{\emph {PreuJoven}}.
\item \textbf{Proyecto de Federación FEUAI, 2009.} \emph{Coordinador General} del primer proyecto de federación en la Universidad que nace luego de la participación de una comisión de alumnos de todos los CCAA en el foro \emph{El Derrumbe de las Cotas} realizado en julio de 2009.
\item \textbf{Cursos de ``Introducción a la Economía'' y ``Política: Instituciones'' (UAI), 2010.} \emph{Ayudante} en ambos cursos dictados por el profesor PhD. Jorge Fábrega Lacoa donde la segunda ayudantía me desempeñé como corrector en el proyecto \emph{Evaluación Electrónica de Ensayos (3E)} de la Universidad.
\item \textbf{Taller de Software, 2010.} \emph{Co-fundador} de taller organizado por alumnos del \emph{Magister en Economía y Políticas Públicas} de la UAI, cuyo objetivo se centró en instruir a sus asistentes en uso de software para investigación académica.
\item \textbf{Taller de Stata MEPP, 2011.} \emph{Ayudante} de curso introductorio a Stata para alumnos del Magíster en Economía y Políticas Públicas de la \emph{Universidad Adolfo Ibáñez}.
\end{itemize}

\section*{Adicional}

\begin{itemize}
\item Inglés \emph{certificado TOIC} con 920/990 puntos --nivel avanzado. Test rendido a finales del 2009, la certificación es válida hasta finales de 2011.
\item Manejo de Ofimática \emph{MS Office 2003-2007} a nivel avanzado, esto es, programación en \emph VBA, \emph{MS Visio} y \emph{MS Access} en redes locales.
\item Manejo de paquetes estadísticos a nivel intermedio: \emph{R-project} y \texttt{STATA}. Además conocimientos en lenguajes \emph{SQL}, \emph{DSPL} y \LaTeX.
\item Coautor en artículo de diario electrónico \emph{El Mostrador} junto con PhD. Jorge Fábrega Lacoa titulado \href{http://www.uai.cl/200911165701/columna-de-opinion/columnas-opinion/reinventar-la-ciudadania-la-otra-campana-que-obama-esta-ganando}{\emph {Reinventar la ciudadanía, la otra campaña que Obama está ganando}} el 10 de Noviembre de 2009.
\end{itemize}

\bigskip

% Footer
\begin{center}
  \begin{footnotesize}
    úlima actualización: \today \\
    \href{\footerlink}{\texttt{\footerlink}}
  \end{footnotesize}
\end{center}

\end{document}