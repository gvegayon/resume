%
% Copyright (C) 2004-2009 Jason Blevins <jrblevin@sdf.lonestar.org>
% http://jblevins.org/projects/cv-template/
%
% You may use use this document as a template to create your own CV
% and you may redistribute the source code freely. No attribution is
% required in any resulting documents. I do ask that you please leave
% this notice and the above URL in the source code if you choose to
% redistribute this file.

\documentclass[letterpaper, 11pt]{article}

\usepackage{hyperref}
\usepackage{geometry}
\usepackage[minbibnames=3,sorting=ydnt,date=comp,isbn=false,doi=false]{biblatex}
\addbibresource{papers.bib}



%\usepackage[sfdefault]{cabin}
%\usepackage[T1]{fontenc}

\usepackage[sfdefault]{overlock} %% Option 'sfdefault' only if the base font of the document is to be sans serif
\usepackage[T1]{fontenc}



% Set your name here
\def\name{George G. Vega Yon}

% Replace this with a link to your CV if you like, or set it empty
% (as in \def\footerlink{}) to remove the link in the footer:
\def\footerlink{https://ggvy.cl}

% The following metadata will show up in the PDF properties
\hypersetup{
  colorlinks = true,
  urlcolor = blue,
  pdfauthor = {\name},
  pdfkeywords = {economics, statistics, mathematics},
  pdftitle = {\name: Curriculum Vitae},
  pdfsubject = {Curriculum Vitae},
  pdfpagemode = UseNone
}

\geometry{
%  body={6.5in, 9in},
  left=1.3cm,
  top=1in,
  right=1.3cm,
  bottom=1in
}

% Customize page headers
\pagestyle{myheadings}
\markright{\name}
\thispagestyle{empty}

% Custom section fonts
%\usepackage{sectsty}
%\sectionfont{\sffamily\mdseries\Large}
%\subsectionfont{\sffamily\mdseries\itshape\large}

% Other possible font commands include:
% rmfamily
% \ttfamily for teletype,
% \sffamily for sans serif,
% \bfseries for bold,
% \scshape for small caps,
% \normalsize, \large, \Large, \LARGE sizes.

% Don't indent paragraphs.
\setlength\parindent{0em}

% Make lists without bullets
\renewenvironment{itemize}{
  \begin{list}{}{
    \setlength{\leftmargin}{0.45cm}
  }
}{
  \end{list}
}

% Para poder poner comandos genericos en tablas (en el inicio del argumento)
\usepackage{array}

\begin{document}

% Place name at left
\part*{\name}

% Alternatively, print name centered and bold:
%\centerline{\huge \bf \name}

%\vspace{0.25in}

\begin{minipage}{0.50\linewidth}
  \begin{tabular}{>{\bfseries}p{.2\linewidth}p{.79\linewidth}}
    Mobile & +1 (six two six) 381 8171 \\
    e-mail & \href{mailto:g.vegayon@gmail.com}{\tt g.vegayon@gmail.com} \\
    website & \href{https://ggvy.cl}{\tt ggvy.cl} \\
    Code & \href{https://github.com/gvegayon}{\tt github.com/gvegayon}\\
    Linkedin & \href{https://www.linkedin.com/in/georgevegayon/}{\tt www.linkedin.com/in/georgevegayon/}
  \end{tabular}
\end{minipage}


\section*{Education}

\begin{itemize}
\item 
{\bf Ph.D. in Biostatistics (with concentration in Stat. Comp.)} University of Southern California (2020). Dissertation \emph{``Essays on Bioinformatics and Social Network Analysis: Statistical and Computational Methods for Complex Systems''}

{\bf M.Sc. in Social Sciences (Economics)} California Institute of Technology (2016)

{\bf Master in Economics and Public Policy}, Adolfo Ib\'a\~nez University (2011)

{\bf BS. in Social Sciences} and {\bf BS. in Business Administration}, Adolfo Ib\'a\~nez University (2010)
\end{itemize}

\section*{Professional Experience}

\begin{itemize}
\item \textbf{University of Southern California, 2015--present} Department of Preventive Medicine \emph{Research Programmer}. A senior staff member at USC's Department of Preventive Medicine, I work closely with both staff and students on various scientific projects. My responsibilities included: implement statistical methods using R/C++, analyze complex data using USC's High-Performance Computing cluster, conducting training sessions on statistical computing, and writing scientific papers.
\item \textbf{Chilean Pension Supervisor, August 2011-- August 2014} Research Division \emph{Analyst}. Statistical and econometric analysis on the Chilean unemployment insurance, statistical software development, serving as a bridge between the IT and Research divisions.
\item \textbf{Nodos Chile Social Network Analysis Ltda., January 2012--January 2014} \emph{Partner}.
Founding partner of one of the first applied SNA Consultancy Entrepreneurship in Chile.
\item \textbf{Adolfo Ib\'a\~nez University, January 2011--June 2012.} School of Government \emph{Adjunct Professor}.
Taught Introductory courses of Economics, Microeconomics and Statistical computing with Stata.
\item \textbf{Chilean Ministry of Social Planning, March 2011--December 201.} Social Programs Monitoring \emph{Analyst}.
Survey and Analysis of the Government social programs supply and supporting the Monitoring Division with the Open-Data Initiative.
\end{itemize}

\section*{Software}

\begin{itemize}
\item[] R, C++, \LaTeX, SQL, XML, regex, Stata+Mata, VBA, Gephi, Pajek, Mathematica, MS Suit, git, unix
\end{itemize}

\section*{Selected Publications and Work in progress}

\nocite{VegaYon2019a,VegaYon2019b,VegaYon2019c,Bell2019,DelaHaye2019,Valente2020,Valente2019,FabregaLacoa2013,vega2015rgexf,vega2019exponential,vega2013abcoptim,vega2020ergmito,vegayon2020aphylo}

\printbibliography[title=\vskip-20pt,omitnumbers=true]

%\subsection*{Conferences and Presentations}
%
%\begin{itemize}
%\item ``PARALLEL: Stata module for parallel computing'', presented at the {\it Stata Conference} (2013), New Orleans, USA.
%\item ``Introducing {\tt rgexf} and how to develop your own R-package'', presetented at the {\it Group of R users in Chile} (2013), Santiago, Chile.
%\item ``Introducing PARALLEL: Stata Module for Parallel Computing'', presented at the {\it Reuni\'on Anual de la Sociedad de Economistas de Chile} (2012), Vi\~na del Mar, Chile.
%\end{itemize}

%\subsection*{Press and others}
%
% \begin{itemize}
%\item
%\href{https://gephi.wordpress.com/2013/02/12/rgexf-an-r-library-to-work-with-gexf-graph-files/}{``rgexf: An R library to work with GEXF graph files''} in {\it Gephi Blog}, published in february 12, 2013.
%\item \href{http://www.elmostrador.cl/opinion/2012/04/04/aborto-en-twitter-un-debate-a-medias/}{``Aborto en Twitter: un debate a medias''} in {\it El Mostrador}, published in april 4, 2012.
%\item \href{http://www.uai.cl/200911165701/columna-de-opinion/columnas-opinion/reinventar-la-ciudadania-la-otra-campana-que-obama-esta-ganando}{``Reinventar la ciudadanía, la otra campaña que Obama está ganando''} in {\it El Mostrador}, published in november 10, 2009.
%\end{itemize}

\section*{Honors and Professional Achievements}

\begin{itemize}
\item \textbf{Book reviewer}: ``Microeconometrics and Matlab: An Introduction'', by Adams, Clarke and Quinn, Oxford University Press, 2015. ``Mastering Gephi Network Visualization'', by Ken Cherven, Packt Publishing, 2015. ``Network Graph Analysis and Visualization with Gephi'', by Ken Cherven, Packt Publishing, 2013.
\item \textbf{Awards} Travel Grant, Society of Young Network Scientist, 2019; Fellowship, California Institute of Technology, 2014; Scholarship, Adolfo Ib\'a\~nes University, 2006.
%\item \textbf{Book reviewer} ``Microeconometrics and MATLAB'', by Abi Adams, Damian Clarke \& Simon Quinn, Oxford University Press, (forthcoming 2015).
\item \textbf{Manuscript reviewer} The Official Journal of The Society for Computational Economics, The R Journal, Social Networks, Journal of Mathematical Sociology, Journal of Open Source Software, SUNBELT Conference (2016), International Conference on Computational Social Science (2019).
\item \textbf{Misc} Founder of the \href{https://www.meetup.com/useRchile/}{R Users Group in Chile (2013)}, co-organizer of the \href{https://socalr.org}{East LA R User Group (LAERUG)}.
%\item \textbf{Honorable Mention} (Posters Sesion. Work: \emph{Introducing Parallel: Stata Module for Parallel Computing}) given by the Chilean Economics Society during its 2012 annual meetting.
%\item \textbf{Honour Scholarship} given by Adolfo Ib\'a\~nez University for completing undergrad and graduate studies (2006-2010).
%\item \textbf{Software workshop, 2010.} \emph{Co-founder} of the workshop thaught by Economics and PP Masters's Students at Adolfo Ib\'a\~nez University in order to introduce grad students into scholar research software (\LaTeX, \emph{R}, Stata, etc.).
\end{itemize}

\bigskip

% Footer
\begin{center}
 \begin{footnotesize}
   last update: \today \\
   \href{\footerlink}{\texttt{\footerlink}}
 \end{footnotesize}
\end{center}

\end{document}

