%
% Copyright (C) 2004-2009 Jason Blevins <jrblevin@sdf.lonestar.org>
% http://jblevins.org/projects/cv-template/
%
% You may use use this document as a template to create your own CV
% and you may redistribute the source code freely. No attribution is
% required in any resulting documents. I do ask that you please leave
% this notice and the above URL in the source code if you choose to
% redistribute this file.

\documentclass[letterpaper, 11pt]{article}

\usepackage{hyperref}
\usepackage{geometry}

%\usepackage[spanish]{babel}
%  \selectlanguage{english}

% Comment the following lines to use the default Computer Modern font
% instead of the Palatino font provided by the mathpazo package.
% Remove the 'osf' bit if you don't like the old style figures.
%\usepackage[T1]{fontenc}
\usepackage[sc,osf]{mathpazo}
%\usepackage[scaled]{uarial}
%\renewcommand*\familydefault{\sfdefault} %% Only if the base font of the document is to be sans serif
%\usepackage[T1]{fontenc}

% Set your name here
\def\name{George G. Vega Yon}

% Replace this with a link to your CV if you like, or set it empty
% (as in \def\footerlink{}) to remove the link in the footer:
\def\footerlink{http://its.caltech.edu/$\sim$gvegayon}

% The following metadata will show up in the PDF properties
\hypersetup{
  colorlinks = true,
  urlcolor = blue,
  pdfauthor = {\name},
  pdfkeywords = {economics, statistics, mathematics},
  pdftitle = {\name: Curriculum Vitae},
  pdfsubject = {Curriculum Vitae},
  pdfpagemode = UseNone
}

\geometry{
%  body={6.5in, 9in},
  left=1.3cm,
  top=1in,
  right=1.3cm,
  bottom=1in
}

% Customize page headers
\pagestyle{myheadings}
\markright{\name}
\thispagestyle{empty}

% Custom section fonts
%\usepackage{sectsty}
%\sectionfont{\sffamily\mdseries\Large}
%\subsectionfont{\sffamily\mdseries\itshape\large}

% Other possible font commands include:
% rmfamily
% \ttfamily for teletype,
% \sffamily for sans serif,
% \bfseries for bold,
% \scshape for small caps,
% \normalsize, \large, \Large, \LARGE sizes.

% Don't indent paragraphs.
\setlength\parindent{0em}

% Make lists without bullets
\renewenvironment{itemize}{
  \begin{list}{}{
    \setlength{\leftmargin}{0.45cm}
  }
}{
  \end{list}
}

% Para poder poner comandos genericos en tablas (en el inicio del argumento)
\usepackage{array}

\begin{document}

% Place name at left
{\huge \name}

% Alternatively, print name centered and bold:
%\centerline{\huge \bf \name}

\vspace{0.25in}

\begin{minipage}{0.45\linewidth}
  \begin{tabular}{>{\bfseries}p{4cm}l}
    Celular & +1 (626) 381 8171 \\
    e-mail & \href{mailto:gvegayon@caltech.edu}{\tt gvegayon@caltech.edu} \\
    website & \href{http://www.its.caltech.edu/~gvegayon}{\tt http://www.its.caltech.edu/$\sim$gvegayon} \\
    C\'odigo & \href{https://github.com/gvegayon}{https://github.com/gvegayon}\\
    & \href{https://bitbucket.org/gvegayon}{https://bitbucket.org/gvegayon}
  \end{tabular}
\end{minipage}

\section*{Educaci\'on}

\begin{itemize}
\item {\bf M.Sc. in Social Sciences} California Institute of Technology (Sep 2015) \\
{\bf Master en Econom\'ia y Pol\'iticas P\'ublicas}, Universidad Adolfo Ib\'a\~nez (2011) \\
{\bf Ingenier\'ia Comercial}, Universidad Adolfo Ib\'a\~nez (2010) \\
{\bf Licenciatura en Ciencias Sociales}, Universidad Adolfo Ib\'a\~nez (2010) 
\end{itemize}

\section*{Experiencia}

\begin{itemize}
\item \textbf{Superintendencia de Pensiones, 2011--2014} Divisi\'on de Estudios \emph{Analista}. An\'alisis estad\'istico y econom\'etrico sobre el seguro de cesant\'ia y el sistema de pensiones, desarrollo de software estad\'istico, trabajando en conjunto con el Departamento de Ingenier\'ia de Sistemas.
\item \textbf{Nodos Chile Social Network Analysis Ltda., 2012--2014} \emph{Socio}.
Socio fundador de una de las primeras empresas consultoras en an\'alisis de redes sociales en Chile.
\item \textbf{Universidad Adolfo Ib\'a\~nez, 2011--2012.} Escuela de Gobierno \emph{Profesor} de Econom\'ia y Estad\'istica Computacional (2012).
Profesor de curso introductorio de econom\'ia, microeconom\'ia, y estad\'istica computacional con Stata.
\item \textbf{Ministerio de Planificac\'ion, 2011.} Monitoreo de Programas Sociales \emph{Analista}.
Coordinandor del proyecto \emph{Banco Integrado de Programas Sociales} y encargado de estrategia de \emph{datos abiertos} en la divisi\'on.
\end{itemize}

\section*{Software}

\begin{itemize}
\item[] Stata (and Mata), R, \LaTeX, SQL, XML, regex, C++ (con R a trav\'es de Rcpp), VB, Gephi, Pajek, Mathematica, MS Access, MS Excel, git, unix,
\end{itemize}

\section*{Investigaci\'on}

\subsection*{Publicaciones y proyectos en desarrollo}
\begin{itemize}
\item F\'abrega, J., \& {\bf Vega, G.} (2013) The impact of television rating over twitter activity: Evidence for Chile based on the Teleton 2012. In {\it Cuadernos.info}, 33. doi:10.7764/cdi.33.533.
%El impacto del rating televisivo sobre la actividad en Twitter: Evidencia para Chile
\item Repetto, A., \& {\bf Vega, G.} (2013). The impact of increasing retirement payroll-tax: Pensions, Salaries and Employment. In {\it Centro de Pol\'iticas Laborales, Adolfo Ib\'a\~nez University}.
%\item {\bf Vega, G}, \& Melse, E. (2013). PARALLEL: multicore computing in Stata.
\item F\'abrega, J., Paredes, P., \& {\bf Vega, G.} (2013). Exploratory Analysis of Television Audience in Twitter. In {\it Informe Anatel 2013, chapter 5, National Television Asociation (ANATEL)}.
%\item ``blopmatching: Implementing bi-level optimization matching algorithm''.
\item Quintanilla, X., Poblete, I., \& {\bf Vega, G.} (2013). Actuarial Analysis on the Unemployment Insurance Funds. In {\it Working Paper N 56, Chilean Pension Supervisor}.
\item {\bf Vega, G.} (2013). Unitary Needed Capital (CNU): Computing and Introducing Stata module {\tt cnu}. In {\it Technical Notes, Chilean Pension Supervisor}.
%\item ``PIB Académico: Din\'amicas de la colaboración acad\'emica en Chile'', {\it Universidad Adolfo Ib\'a\~nez}
%\item ``Formaci\'on de la Opini\'on P\'ublica en los Medios Sociales'' en {\it Universidad Adolfo Ib\'a\~nez}.
\end{itemize}

\subsection*{Componentes de Software destacado}
\begin{itemize}
\item Rau, T., {\bf Vega, G.} \& Espinoza, D. ``blopmatching: Implementing a New Matching Estimator based on a Bi-Level Optimization Problem'', (2014). Developed for Diaz, Rau and Rivera (RESTAT, 2015)
\item {\bf Vega, G.}, \& Mu\~noz, E. (2013). ABCoptim: An Implementation of the Artificial Bee Colony (ABC) Optimization algorithm. In {\it The Comprehensive R Archive Network}.
\item {\bf Vega, G.} (2012). PARALLEL: Stata module for Parallel Computing. In {\it Statistical Software Components, Boston College Department of Economics}.
\item {\bf Vega, G.}, F\'abrega, J., \& Kunst, J. (2012). rgexf: An R package to build gexf graph files. In {\it The Comprehensive R Archive Network}.
\item {\bf Vega, G.}, \& Fajnzylber, E. (in development). NNMATCH2: A faster implementation of the Nearest Neighbour Matching Estimator.
\item {\bf Vega, G.} (2012). googlePublicData: An R package to build Google's Public Data Explorer DSPL Metadata files, In {\it The Comprehensive R Archive Network}.
\end{itemize}

%\subsection*{Conferences and Presentations}
%
%\begin{itemize}
%\item ``PARALLEL: Stata module for parallel computing'', presented at the {\it Stata Conference} (2013), New Orleans, USA.
%\item ``Introducing {\tt rgexf} and how to develop your own R-package'', presetented at the {\it Group of R users in Chile} (2013), Santiago, Chile.
%\item ``Introducing PARALLEL: Stata Module for Parallel Computing'', presented at the {\it Reuni\'on Anual de la Sociedad de Economistas de Chile} (2012), Vi\~na del Mar, Chile.
%\end{itemize}

%\subsection*{Press and others}
%
% \begin{itemize}
%\item
%\href{https://gephi.wordpress.com/2013/02/12/rgexf-an-r-library-to-work-with-gexf-graph-files/}{``rgexf: An R library to work with GEXF graph files''} in {\it Gephi Blog}, published in february 12, 2013.
%\item \href{http://www.elmostrador.cl/opinion/2012/04/04/aborto-en-twitter-un-debate-a-medias/}{``Aborto en Twitter: un debate a medias''} in {\it El Mostrador}, published in april 4, 2012.
%\item \href{http://www.uai.cl/200911165701/columna-de-opinion/columnas-opinion/reinventar-la-ciudadania-la-otra-campana-que-obama-esta-ganando}{``Reinventar la ciudadanía, la otra campaña que Obama está ganando''} in {\it El Mostrador}, published in november 10, 2009.
%\end{itemize}

\section*{Honores y Logros}

\begin{itemize}
\item \textbf{Revisi\'on de libro} ``Microeconometrics and Matlab: An Introduction'', by Adams, Clarke and Quinn, Oxford University Press, Forthcoming 2015.
\item \textbf{Revisi\'on de libro} ``Mastering Gephi Network Visualization'', by Ken Cherven, Packt Publishing, 2015.
\item \textbf{Beca} California Institute of Technology, for Ph.D. Programme in Social Sciences.
%\item \textbf{Book reviewer} ``Microeconometrics and MATLAB'', by Abi Adams, Damian Clarke \& Simon Quinn, Oxford University Press, (forthcoming 2015).
\item \textbf{Ad hoc Peer-reviewer} (parallel computing) of the Official Journal of The Society for Computational Economics (January 2013).
\item \textbf{Revisi\'on de libro} ``Network Graph Analysis and Visualization with Gephi'', by Ken Cherven, Packt Publishing, 2013.
\item \textbf{Fundador del Grupo de Usuarios de R en Chile} Starting in January 2013.
%\item \textbf{Honorable Mention} (Posters Sesion. Work: \emph{Introducing Parallel: Stata Module for Parallel Computing}) given by the Chilean Economics Society during its 2012 annual meetting.
\item \textbf{Beca de Honor} Universidad Adolfo Ib\'a\~nez para completar estudios de pregrado y postgrado (2006-2010).
\item \textbf{Director y co-fundador del proyecto ``Educaci\'on Joven'', 2008--2009.} Programa de tutor\'ias PSU para estudiantes de recursos escasos.
%\item \textbf{Software workshop, 2010.} \emph{Co-founder} of the workshop thaught by Economics and PP Masters's Students at Adolfo Ib\'a\~nez University in order to introduce grad students into scholar research software (\LaTeX, \emph{R}, Stata, etc.).
\end{itemize}

%\bigskip

% Footer
%\begin{center}
%  \begin{footnotesize}
%    last update: \today \\
%    \href{\footerlink}{\texttt{\footerlink}}
%  \end{footnotesize}
%\end{center}

\end{document}

